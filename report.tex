\documentclass[14pt,a4paper]{extarticle}
\usepackage[utf8]{inputenc}
\usepackage[russian]{babel}
\usepackage{setspace}
\usepackage{geometry}
\usepackage{setspace}
\usepackage{titlesec}
\usepackage{graphicx}


\geometry{left=2.5cm,right=2.5cm,top=2cm,bottom=2cm}
\onehalfspacing


\titleformat{\section}{\normalfont\fontsize{14}{0}\bfseries}{\thesection}{}{}
\titleformat{\subsection}{\normalfont\fontsize{14}{0}\bfseries}{\thesubsection}{}{}

\titlespacing*{\section}
    {0pt}
    {15pt}                   
    {10pt}    

\begin{document}

\begin{titlepage}
    \begin{center}
        
        \textbf{МОСКОВСКИЙ АВИАЦИОННЫЙ ИНСТИТУТ} \\
        \textbf{(НАЦИОНАЛЬНЫЙ ИССЛЕДОВАТЕЛЬСКИЙ УНИВЕРСИТЕТ)}
        
        \vspace{1cm}
        
        Институт №8 «Компьютерные науки и прикладная математика» \\
        Кафедра 806 «Вычислительная математика и программирование»
        
        \vspace{4cm}
        
        \textbf{Лабораторная работа №1} \\
        \textbf{по курсу «Операционные системы»}
        
        \vspace{8cm}
        
        \begin{flushright}
            Выполнила: Власова Е.Р. \\
            Группа: М8О-208БВ-24 \\
            Преподаватель: А. Ядров
        \end{flushright}
        
        \vfill
        
        Москва, 2025
        
    \end{center}
\end{titlepage}


\section*{Условие:}

Составить и отладить программу на языке Си, осуществляющую работу с процессами и взаимодействие между ними в одной из двух операционных систем. В результате работы программа (основной процесс) должен создать для решение задачи один или несколько дочерних процессов. Взаимодействие между процессами осуществляется через системные сигналы/события и/или каналы (pipe). Необходимо обрабатывать системные ошибки, которые могут возникнуть в результате работы.
\section*{Цель работы:}
 Приобретение практических навыков в управлении процессов в ОС, Обеспечение обмена данных между процессами посредством каналов.
\section*{Задание:}
Родительский процесс создает дочерний процесс. Первой строкой пользователь в консоль родительского процесса вводит имя файла, которое будет использовано для открытия File с таким именем на запись. Перенаправление стандартных потоков ввода-вывода показано на картинке выше. Родительский и дочерний процесс должны быть представлены разными программами. Родительский процесс принимает от пользователя строки произвольной длины и пересылает их в pipe1. Процесс child проверяет строки на валидность правилу. Если строка соответствует правилу, то она выводится в стандартный поток вывода дочернего процесса, иначе в pipe2 выводится информация об ошибке. Родительский процесс полученные от child ошибки выводит в стандартный поток вывода. Правило проверки: строка должна начинаться с заглавной буквы.
\section*{Вариант:}
15

\section*{Метод решения:}
Общее описание алгоритма решения задачи, архитектуры программы и т. п. Полностью расписывать алгоритмы необязательно, но в общих чертах описать нужно. Приветствуются ссылки на внешние источники, использованные при подготовке (книги, интернет-ресурсы).

\section*{Описание программы:}
Разделение по файлам, описание основных типов данных и функций. Обязательно написать используемые системные вызовы.

\section*{Результаты:}
Описать полученное решение, ключевые особенности. В ЛР с потоками привести график замеров.

\section*{Выводы:}
Приобрела практические навыкы в управлении процессов в ОС и узнала про Обеспечение обмена данных между процессами посредством каналов. Осознала суть понятий "поток" и "процесс", поняла границы их применения в рамках ОС. 
\newpage
\section*{Исходная программа:}

\begin{verbatim}
#include <iostream>

int main() {
    std::cout << "Hello, world!" << std::endl;
    return 0;
}
\end{verbatim}


\end{document}
